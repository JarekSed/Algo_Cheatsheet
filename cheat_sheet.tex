\documentclass[12pt,twocolumn]{article}
\usepackage{url,graphicx,tabularx,array,amsmath}
\usepackage{enumitem}
\setlist{nolistsep} %-- don't have lists take up shitloads of room
\setlength{\parskip}{1ex} %--skip lines between paragraphs
\setlength{\parindent}{0pt} %--don't indent paragraphs
\usepackage[
top    = 0.75cm,
bottom = 0.50cm,
left   = 0.50cm,
right  = 0.50cm]{geometry}

%-- Commands for header
\renewcommand{\title}[1]{\textbf{#1}}
\renewcommand{\line}{\begin{tabularx}{\textwidth}{X>{\raggedleft}X}\hline\\\end{tabularx}\\[-0.5cm]}
\newcommand{\leftright}[2]{\begin{tabularx}{\textwidth}{X>{\raggedleft}X}#1%
& #2\\\end{tabularx}}

%\linespread{2} %-- Uncomment for Double Space
\begin{document}

\section{Master Theorem}
Given $T(1) = c$.\\
\[T(n) = a * T\left(\frac{n}{b}\right) + d*f(n), \mbox{for multiplicative f}\]
\begin{align}
    \Theta(n^{\log_{b} a}) \mbox{ if }a \ge f(b)\\
    \Theta(f(n)) \mbox{ if } f(b) > a\\
    \Theta(n^{\log_{b} a}* \log n) \mbox{ if } f(b) = a
\end{align}
\section{Multiplicative}
$f(x)$ is multiplicative iff $f(xy) =  f(x) * f(y)$.\\
For example, $f(x) = x^2$ is multiplicative.\\
$f(x) = 2x$ is not multiplicative.

\section{Big O}

$ f(x) \mbox{ is } O(g(x)) => \lim_{x \to \infty}{\frac{f(x)}{g(x)}} \le c \mbox{ for some constant c}$\\
$ f(x) \mbox{ is } \Omega(g(x)) => \lim_{x \to \infty}{\frac{f(x)}{g(x)}} \ge c \mbox{ for some constant c}$\\
$ f(x) \mbox{ is } \Theta(g(x)) => \lim_{x \to \infty}{\frac{f(x)}{g(x)}} = c \mbox{ for some constant c}$\\

\section{Random Facts}
\begin{enumerate}
    \item You can multiply in $\Theta(n^{\log_2 3})$.
    \item Bubble sort is worst case $\Theta(n^2)$ and best case $\Theta(n)$ and average case $\Theta(n^2)$ - Stable, In place
    \item Merge sort is $\Theta(n \log n)$ for all cases. - Stable, Not in place
    \item Quick sort is worst case $\Theta(n^2)$ and best case $\Theta(n \log n)$. - Not stable, In place
    \item Randomized quick sort is expected $\Theta(n \log n)$. Same as quicksort
    \item You can sort (comparitively) no faster than $O(n \log n)$
        \begin{enumerate}
            \item A decision tree for sorting has $>= n!$ leaves. This is because we have at least $n!$ different inputs for a sorting algorithm (one for each permutation of input).
            \item Minimum average depth of a tree with $k$ leaves is $\Omega(\log k)$. So if a decision tree has $n!$ leaves, it must have $\log n!$ depth, which is $\Omega(n \log n)$.
        \end{enumerate}
    \item Non Comparision Sorting
        \begin{enumerate}
            \item Bucket Sort/Bin Sort - hash your key to a bucket. Add it to the linked list for the hashed bucket. Now iterate through buckets, and print out each linked list in order. Requires a bounded input set. $\Theta(n)$ where $n$ is the number of possible inputs. - Stable, Not in place
            \item Radix Sort - Bin Sort based on the least significant digit until you run out of digits. Requires a bounded input set. If we sort $n$ numbers in base $m$ and they are each $k$ digits long, it takes $O(k(n + m))$ - Stable, Not in place
        \end{enumerate}
\end{enumerate}

\section{QuickSelect}
This takes in an array, and finds the $k$th largest element in the array.
\begin{enumerate}
    \item Split array into subarrays of size 5. Find the median of each subarray. This takes $O(n)$ comps.
    \item Find the median of the medians of all the subarrays recursively. This median must be bigger than $\frac{3n}{10}$ elements and smaller than $\frac{n}{10}$ elements.
    \item This tells us $T(n) = T(\frac{n}{5}) + T(\frac{7n}{10}) + n = \Theta(n)$
\end{enumerate}

\section{Graphs}
\begin{enumerate}
    \item A simple path is a path without cycles
    \item A subgraph is strongly connected if you can get from all nodes in it to all other nodes.
    \item A subgraph is weakly connected if its directed, but would be strongly connected if it was undirected.
    \item Trees have $n-1$ edges. Have at most $n-1$ leaves. Has an average depth of $\log_2 n$
    \item Adjacency Matrix is used to represent dense graphs, Adjacency linked lists are used to represent sparse graphs.
\end{enumerate}



\subsection{Minimal Spanning Tree}
Greedy Algorithm
\begin{enumerate}
    \item Initialize all nodes to be infinite distance so far $O(n)$
    \item Assume we have a MST so far.
    \item Add the cheapest edge to the MST that adds a node we haven't seen before.
    \item loop over steps 2-3 $O(n)$
\end{enumerate}
$O(n^2)$ total\\


Kruskall's Algorithm
\begin{enumerate}
    \item Sort edges by weight. $O(logn^m) = O(m\log{n})$
    \item Add an edge to MST if it connects two subgraphs that aren't already connected. $O(n\log{n})$ for
        unions ($O(\log{n}$ and we do it $n$ times) and $O(m)$ for finds (O(1) and we do it $m$ times), so total for this step is $O(n\log{n}+m)$
\end{enumerate}

Use Union Find for Step \#2. Complexity $m \log n$, where $m$ is the number of edges.

\subsection{Dijkstras Algorithm}
\begin{enumerate}
    \item Assume we have shortest paths for everything we've seen so far
    \item Choose the edge that minimizes the full path to a vertex we haven't seen yet.
    \item $O(n^2)$ to find paths from one node to all others, $O(n^3)$ to find for all nodes.
    \item Basically, just like Greedy, but we minimize path to node + new edge, not just new edge
\end{enumerate}

\subsection{Floyd's Algorithm}
\begin{enumerate}
    \item Floyd's algorithm computes all pairs shortest path given a graph without negative cycles.
    \item It runs in $O(n^3)$ time, takes $O(n^2)$ space.
    \item It sets all distances to $\inf$. Visits each node, and updates the distance to the node if we have a shorter path.
\end{enumerate}

Clique is a subgraph in a graph in which all nodes are connected to each other directly.
Independent set is a subgraph in a graph in which no nodes are connected to each other at all.

MAX-CLIQUE - Iterate from i=n to 1 and call Clique(G, i) on it
FIND-CLIQUE - remove nodes, and check to see if Clique still exists

LCA - Lowest Common Ancestor
RMQ - Range Minimum Query

Euler Tour - Write down each node as you visit it during an in order traversal. Size 2n - 1.
Cartesian Trees - An in order traversal of the tree should return the input array. Every node must be smaller than its children. Construction takes $O(n)$.

RMQ \& LCA - can be solved in $O(n)$ preprocess $O(1)$ query. LCA(Cartesian Tree(Array)) = RMQ(Array)

\subsection{DFS}
\begin{enumerate}
    \item Takes $O(n+m)$ time.
    \item Finds connected components.
    \item Tells you if you have a DAG (Directed Acyclic Graph) - If the DFS contains back edges, then you have a cycle.
    \item Top sort - Number edges from $n$ to 1 as you finish DFS at a vertex.
    \item Strongly Connected Components - Top Sort, reverse edges, run DFS in order of Top Sort.
\end{enumerate}

\subsection{Integers}

Interface for these structures is as follows
\begin{enumerate}
    \item Insert(x $\in$ U)
    \item Delete(x $\in$ S)
    \item Predecessor(x $\in$ U)
    \item Successor(x $\in$ U)
    \item w here is the number of bits in your key (32 for integers).
    \item N here is the maximum range of your key. ($2^{32}$ for integers).
    \item n here is the number of elements in your data structure. (like $O(n)$ time to insert into a linked list).
\end{enumerate}

Van emde Boas Trees - Does everything in $O(\log w)$ or $O(\log \log N)$ time.

Fusion Trees - Do everything in $O(\log_{w}{n})$ time or $O(\frac{\log{n}}{\log{w}})$ time.

You can sort using these in $n * $min$(\log w, \log_{w}{n})$ time.

Both take $O(N)$ space.

A van emde boas tree that stores $N$ things contains
\begin{enumerate}
    \item min and max of its elements.
    \item $\sqrt{N}$ van emde boas trees that each store $\sqrt{N}$ things - We will call these the buckets.
    \item A summary tree stores $\sqrt{N}$ things. These things are the indexes of the above that are nonempty.
\end{enumerate}

To insert/delete a number, you split up the higher and lower order bits of the input. You use the high order bits to determine which bucket to put the number in. We will then recurisvely insert/delete the lower-order bits into that bucket (tree), and update min/max.

To successor a number, you compare to min/max. If it's between min and max, you split the bits and index with high order bits into a bucket. Compare to min/max of the bucket. If it's between min/max, recursively call successor on bucket using low order bits. If it's less than min, return min. If it's greater than max, recursively call the summary widget using the high order bits to find the next non empty bucket and return its min.

The trick here is that if you make one recursive call on input of size $O(\sqrt{N}$, you get $O(\log \log N)$ time.

\subsection{NP Completeness}
\begin{enumerate}
    \item A decision problem is in NP if it has a polynomial length proof and is checkable in polynomial time.
\end{enumerate}



\end{document}
